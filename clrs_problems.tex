\documentclass[oneside]{book}
\usepackage{geometry}
\geometry{letterpaper}
\usepackage[parfill]{parskip} 
\usepackage{graphicx}
\usepackage{setspace}
\usepackage{booktabs}
\usepackage{fancyhdr}

\usepackage{amsthm, amsmath, amssymb, amsfonts}
\theoremstyle{definition}
\newtheorem{problem}{Problem}[section]
\newtheorem{definition}{Definition}[chapter]
\newtheorem{rem}{Remark}[chapter]

\theoremstyle{plain}
\newtheorem{thm}{Theorem}[chapter]
\newtheorem{lem}{Lemma}[chapter]

\pagestyle{fancy}
\fancypagestyle{plain}{}

\usepackage[
	backend=biber,
	style=ieee
]{biblatex}
\addbibresource{clrs_problems.bib}

\usepackage{makeidx}
\makeindex
\usepackage[acronym]{glossaries}
\makeglossaries

\usepackage[toc,page]{appendix}

\usepackage[
	colorlinks,
	linkcolor=blue,
	citecolor=blue,urlcolor=blue
]{hyperref} 

\usepackage[dvipsnames]{xcolor}
\usepackage{listings}
\lstdefinestyle{mystyle}{
    basicstyle=\ttfamily\footnotesize,
    breakatwhitespace=false,         
    breaklines=true,                 
    captionpos=b,                    
    keepspaces=true,                 
    numbers=left,                    
    numbersep=5pt,                  
    showspaces=false,                
    showstringspaces=false,
    showtabs=false,                  
    tabsize=2,
    frame=single
}
\lstset{
	style=mystyle,
	language=C++,
	rangeprefix=//---- ,
	rangesuffix=----,
	includerangemarker=false,
	keywordstyle=\color{blue},
	commentstyle=\color{ForestGreen},
	stringstyle=\color{red},
	morekeywords={true, false, nullptr}
}

\newcommand{\seq}[1]{\left\langle #1 \right\rangle} 

\begin{document}

% set the footer 
% clear the footer
\fancyhf{} 
% remove the horizontal line below the header. Remove the line below if you want header
\renewcommand{\headrulewidth}{0pt}
\fancyfoot[L]{CLRS Notes and Problems}
\fancyfoot[C]{\thepage}
%\fancyfoot[R]{}

%\maketitle
\frontmatter
\begin{titlepage}
	\vspace{1 in}
	\centering
	{\LARGE {CLRS Notes and Problems}\par}
	\vspace{1 cm}
	Author\par 
	Mike Conlen \texttt{mike@conlen.org}\par
	\vspace{1 cm}
	\today
\end{titlepage}

\thispagestyle{empty}
\textbf{Author}: Mike Conlen \\
% Contributors go in alphabetical order by last name

% this alternates pipe and the letter L in lower case. In many fonts these look the same 
\begin{table}[ht]
	\begin{tabular}{l l l}
		\toprule
		Version & Date & Notes \\
		\hline
		&& \\
		\bottomrule
	\end{tabular}
	\caption{Change History}\label{tab:history}
\end{table}

\tableofcontents
\lstlistoflistings

\mainmatter
\begin{spacing}{1.618}
\chapter{The Role of Algorithms in Computing}
\section{Algorithms}
\subsection{Notes}
\begin{definition}[algorithm]\index{algorithm}
	A well-defined computational procedure that takes some value or set of values as input and produces some value or set of values as output in a finite amount of time.
\end{definition}

\begin{definition}[sorting problem]\index{sorting problem}
	~\\
	\textbf{Input: } A sequence of $n$ numbers $\seq{a_1, a_2, \dots, a_n}$ \\
	\textbf{Output:} A permutation $\seq{a'_k}$ of the sequence $\seq{a_k}$ such that $a'_m \leq a'_n$ for all $m < n$. 
\end{definition}

\begin{definition}[problem instance]\index{problem!instance}
	A computational problem and a specific input.
\end{definition}

\begin{definition}[Halt]\index{halt}
	An algorithm \emph{halts} if it solves a computational problem in finite time
\end{definition}

\begin{definition}[correct]\index{correct}
	An algorithm is \emph{correct} if for every problem instances the algorithm halts and outputs the correct solution to the instance.
\end{definition}

\begin{definition}[data structure]\index{data structure}
	A way to store and organize data in order to facilitate access and modifications.
\end{definition}

\begin{definition}[online algorithm]\index{algorithm!online}
	An algorithm which receives input over time.
\end{definition}

\subsection{Problems}

\begin{problem}
Describe your own real-world example that requires sorting. Describe one that requires finding the shortest distance between two points.
\begin{proof}[Solution] \ \par
	\begin{enumerate}
		\item Putting cards into order in a bridge hand. 
		\item Using the triangle inequality. 
	\end{enumerate}
\end{proof}
\end{problem}

\begin{problem}
	Other than speed, what other measures of efficiency might you need to consider in a real-world setting?
	\begin{proof}[Solution]
		Minimizing waste from raw materials when deciding how to cut pieces from a a set of larger pieces of material. 
	\end{proof}
\end{problem}

\begin{problem}
	Select a data structure that you have seen, and discuss its strengths and limitations.
	\begin{proof}[Solution]
		An array. When adding elements into the middle of an array you may need to copy the entire array into a larger area of memory. 
	\end{proof}
\end{problem}

\begin{problem}
	How are the shortest-path and traveling-salesperson problems given above similar?
	How are they different?
	\begin{proof}[Solution]
		The shortest path problem requires only the start and end points be visited within a graph. The traveling salesperson problem requires that a larger sets of nodes be visited and the first and last nodes are the same. 
	\end{proof}
\end{problem}

\begin{problem}
		Suggest a real-world problem in which only the best solution will do. Then come up with one in which the best solution is good enough.
	\begin{proof}[Solution]
		Sorting words for a dictionary requires only the best solution since an approximate solution makes it difficult for a user to find the words they are searching for. Finding a driving route requires only an approximate solution since the user ultimately arrives at their destination. 
	\end{proof}
\end{problem}

\begin{problem}
	Describe a real-world problem in which sometimes the entire input is available before you need to solve the problem, but other times the input is not entirely available in advance and arrives over time.
	\begin{proof}[Solution]
		A problem where the entire input is available before solving is building a lego kit. A problem where the entire input isn't available is creating online dictionaries as new words are created and added to an existing sorted set of words on a regular basis. 
	\end{proof}
\end{problem}

\section{Algorithms as Technology}

\subsection{Notes}

\subsection{Problems}

\begin{problem}
	Give an example of an application that requires algorithmic content at the application level, and discuss the function of the algorithms involved.
	\begin{proof}[Solution]
		Route planning software for a package delivery service requires an algorithm to plan the route of each truck. The algorithm would ensure that time isn't wasted in delivery and fuel is conserved. It would output the route for each driver on their shift. 
	\end{proof}
\end{problem}

\begin{problem}
	Suppose that for inputs of size n on a particular computer, insertion sort runs in $8n^2$ steps and merge sort runs in $64n\lg{n}$ steps. For which values of n does insertion sort beat merge sort?
	\begin{proof}[Solution]
		Assuming steps are equally expensive we want to find the $n\in\mathbb{N}$ such that $8n^2 < 64n\lg{n}$. Since $n>=0$ this implies that $8n < 64\lg{n}\implies n < 8\lg{n}$. 
		
		When $n=0$ the right side is undefined and when $n=1$ the right side is 0; however for $n=2$ we have $2<8$ and the expression is true while $n<44$; therefore $\{n\in\mathbb{N}\mid n\in[2, 44]\}$
	\end{proof}
\end{problem}

\begin{problem}
	What is the smallest value of n such that an algorithm whose running time is $100n^2$ runs faster than an algorithm whose running time is $2^n$ on the same machine
	\begin{proof}[Solution]
		For $n\in \mathbb{N}$ the smallest value is $n=15$
	\end{proof}	
\end{problem}

\section{Chapter 1 Problems}

\begin{problem}[1-1]
	For each function $f(n)$ and time $t$ in the following table, determine the largest size $n$ of a problem that can be solved in time $t$, assuming that the algorithm to solve the problem takes $f(n)$ microseconds. 
	\begin{proof}[Solution]
		We note there are 1,000,000 microseconds in a second. We note 1 minute has 60 seconds, 1 hour has 3,600 seconds, 1 day has 86,400 seconds, 1 month has 30 days or 2,592,000 seconds, 1 year has 365 days or 31,536,000 seconds and 1 century has   3,153,600,000 seconds. In each case we wish to find the largest $n$ such that $f(n)$ is less than $t$ in microseconds. 
		
		\begin{tabular}{l|c|c|c|c|c|c|c|}
			 & 1 second & 1 minute & 1 hour & 1 day & 1 month & 1 year & 1 century \\
			 \hline
			 $\lg{n}$ & $2^{10^6}$ & $2^{6\cdot 10^7}$ & $ 2^{3.6 \cdot 10^8}$ & $2^{8.64 \cdot 10^9}$ & $2^{2.592\cdot 10^{12}}$ & $2^{3.1536\cdot 10^{13}}$ & $2^{3.1536 \cdot 10^{15}}$ \\
			 \hline
			 $\sqrt{n}$ & $10^{12}$ & $36 \cdot 10^{14}$ & $1.296 \cdot 10^{19}$ & $7.464\cdot 10^{21}$ & $6.718\cdot 10^{24}$ & $9.945\cdot 10^{26}$ & $9,945\cdot 10^{30}$ \\
			 \hline 
			 $n$ & $10^6$ & $6 \cdot 10^7$ & $3.6\cdot 10^9$ & $8.64\cdot 10^{10}$ & $2.59\cdot 10^{12}$ & $3.15\cdot 10^{13}$ & $3.15\cdot 10^{15}$ \\
			 \hline
			 $n~\lg{n}$ & $62746$ & $2.80\cdot 10^6$ & $1.33\cdot 10^8$ & $2.76\cdot 10^9$ & $7.19\times 10^{10}$ & $7.98\cdot 10^{11}$ & $6.86\cdot 10^{13}$ \\
			 \hline
			 $n^2$ & $1000$ & $7745$ & $60000$ & $293939$ & $1.61\cdot 10^6$ & $5.62\cdot 10^6$ & $5.62\cdot 10^7$ \\
			\hline
			 $n^3$ & $100$ & $391$ & $ 1532$ & $4220$ & $13736$ & $27636$ & $14664$\\
			\hline
			 $2^n$ & $19$ & $25$ & $31$ & $36$ & $41$ & $ 44$ & $ 51$ \\
			\hline
			 $n!$ & $9$ & $11$ & $12$ & $13$ & $15$ & $16$ & $17$\\
			\hline
		\end{tabular}
	\end{proof}
\end{problem}

\chapter{Getting Started}
\section{Insertion Sort}

\begin{rem}\index{insertion sort}
	Insertion sort is a solution to the sorting problem. An implementation is given in Listing \ref{listing:insertion_sort}.
\end{rem}

\begin{definition}[key]\index{key}
	The values on which an algorithm acts; e.g. the values being sorted in a sorting algorithm. 
\end{definition}

\begin{definition}[satellite data]\index{satellite data}\index{data!satellite}
	Data associated with a key. 
\end{definition}

\begin{definition}[record]\index{record}
	A key and the associated satellite data.
\end{definition}

\begin{minipage}{\textwidth}
	\lstinputlisting[
		language=C++, 
		firstnumber=0,
		caption=Insertion Sort,
		linerange=begin:insertion_sort-end:insertion_sort
	]{code/src/insertion-sort.hpp}\label{listing:insertion_sort}
\end{minipage}

\begin{rem}
	There are two things we prove about the algorithm in Listing \ref{listing:insertion_sort}. The first is that the iterators are valid at each step of the program. If the iterators become invalid we may trigger undefined behavior within the compiler and the program becomes incorrect. The second is that the algorithm does what we claim it does, which is sort the range of the container defined by \texttt{[begin, end)}. 
\end{rem}

\begin{thm}
	In the implementation of insertion sort in Figure \ref{listing:insertion_sort} the iterators remain inside $[\texttt{begin}, \texttt{end}]$, avoiding undefined behavior assuming \texttt{begin} and \texttt{end} are valid iterators for elements of the same container. 
	\begin{proof}
		The implementation uses the following iterators; \texttt{begin}, \texttt{end}, \texttt{current}, \texttt{j} and \texttt{rend}. The iterators \texttt{begin} and \texttt{end} are never changed; therefore remain inside the range. 
		
		The iterator \texttt{current} is created and modified in line 6. The initial value is \texttt{begin + 1}. Since \texttt{begin} is less than \texttt{end} (line 4) we know that \texttt{begin + 1} is not greater than \texttt{end}. The for loop increments \texttt{current} until \texttt{current} is not less than \texttt{end} and so, as long as \texttt{current} was valid at the first iteration it remains valid in each iteration of the loop. Finally when \texttt{current} is not less than \texttt{end} it is equal to \texttt{end}, which is valid.
		
		The iterator \texttt{rend} is set to \texttt{begin} and decremented. NB that reverse iterators are actually pointing to the subsequent element such that when \texttt{rend} is decremented it's internally pointing to the element at \texttt{begin} and is thus valid\footnote{to put it another way for a forward iterator \texttt{i} and the reverse iterator \texttt{r} made from \texttt{i} the following relationship holds \texttt{\&*r == \&*(i - 1)} \cite{reverse_iterator} }. 
		
		The iterator \texttt{j} is set to a reverse iterator of \texttt{current} at line 8, which is valid. NB: \texttt{\&*j == \&*(current-1)} \cite{reverse_iterator}. We know \texttt{j} is valid when initialized by the validity of \texttt{current} and \texttt{j} points to the element to the left of \texttt{current} and \texttt{(j-1)} points to the element under \texttt{current}. We loop while \texttt{j <= rend} incrementing \texttt{j} each time, which moves it to the left since it is a reverse iterator. Recall \texttt{rend} points to the element under \texttt{begin}. This implies that \texttt{j} is valid as in each iteration \texttt{j $\in$ [begin, current)}. Moreover, since \texttt{j} is built from \texttt{current} which is strictly less than \texttt{end} the iterator \texttt{(j-1)} points to an element in the range \texttt{(begin, current]}. This shows that both \texttt{j} and \texttt{j-1} are valid throughout the loop. When the loop ends if \texttt{j == rend} we have that \texttt{j} is valid and \texttt{(j-1)} is in the range \texttt{(begin, current]} which is in the range \texttt{(begin, end]}.  
		
		We are left to prove that in line 12 that \texttt{(j-1) $\in$ [begin, end)}. This follows from the above with the caveat that \texttt{current} is not \texttt{end}, but this follows from the condition in line 6. 
	\end{proof}
\end{thm}



\appendix
\chapter{Appendix One}

\begin{problem}[]
	\begin{proof}[Solution]

	\end{proof}
\end{problem}

\end{spacing}
\backmatter

\addcontentsline{toc}{chapter}{Acronyms}
\printglossary[type=\acronymtype]
\clearpage

\addcontentsline{toc}{chapter}{Glossary}
\printglossary
\clearpage

\addcontentsline{toc}{chapter}{Bibliography}
\printbibliography
\clearpage

\printindex
%
\end{document}  